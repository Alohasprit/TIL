
% Default to the notebook output style

    


% Inherit from the specified cell style.




    
\documentclass[11pt]{article}

    
    
    \usepackage[T1]{fontenc}
    % Nicer default font (+ math font) than Computer Modern for most use cases
    \usepackage{mathpazo}

    % Basic figure setup, for now with no caption control since it's done
    % automatically by Pandoc (which extracts ![](path) syntax from Markdown).
    \usepackage{graphicx}
    % We will generate all images so they have a width \maxwidth. This means
    % that they will get their normal width if they fit onto the page, but
    % are scaled down if they would overflow the margins.
    \makeatletter
    \def\maxwidth{\ifdim\Gin@nat@width>\linewidth\linewidth
    \else\Gin@nat@width\fi}
    \makeatother
    \let\Oldincludegraphics\includegraphics
    % Set max figure width to be 80% of text width, for now hardcoded.
    \renewcommand{\includegraphics}[1]{\Oldincludegraphics[width=.8\maxwidth]{#1}}
    % Ensure that by default, figures have no caption (until we provide a
    % proper Figure object with a Caption API and a way to capture that
    % in the conversion process - todo).
    \usepackage{caption}
    \DeclareCaptionLabelFormat{nolabel}{}
    \captionsetup{labelformat=nolabel}

    \usepackage{adjustbox} % Used to constrain images to a maximum size 
    \usepackage{xcolor} % Allow colors to be defined
    \usepackage{enumerate} % Needed for markdown enumerations to work
    \usepackage{geometry} % Used to adjust the document margins
    \usepackage{amsmath} % Equations
    \usepackage{amssymb} % Equations
    \usepackage{textcomp} % defines textquotesingle
    % Hack from http://tex.stackexchange.com/a/47451/13684:
    \AtBeginDocument{%
        \def\PYZsq{\textquotesingle}% Upright quotes in Pygmentized code
    }
    \usepackage{upquote} % Upright quotes for verbatim code
    \usepackage{eurosym} % defines \euro
    \usepackage[mathletters]{ucs} % Extended unicode (utf-8) support
    \usepackage[utf8x]{inputenc} % Allow utf-8 characters in the tex document
    \usepackage{fancyvrb} % verbatim replacement that allows latex
    \usepackage{grffile} % extends the file name processing of package graphics 
                         % to support a larger range 
    % The hyperref package gives us a pdf with properly built
    % internal navigation ('pdf bookmarks' for the table of contents,
    % internal cross-reference links, web links for URLs, etc.)
    \usepackage{hyperref}
    \usepackage{longtable} % longtable support required by pandoc >1.10
    \usepackage{booktabs}  % table support for pandoc > 1.12.2
    \usepackage[inline]{enumitem} % IRkernel/repr support (it uses the enumerate* environment)
    \usepackage[normalem]{ulem} % ulem is needed to support strikethroughs (\sout)
                                % normalem makes italics be italics, not underlines
    

    
    
    % Colors for the hyperref package
    \definecolor{urlcolor}{rgb}{0,.145,.698}
    \definecolor{linkcolor}{rgb}{.71,0.21,0.01}
    \definecolor{citecolor}{rgb}{.12,.54,.11}

    % ANSI colors
    \definecolor{ansi-black}{HTML}{3E424D}
    \definecolor{ansi-black-intense}{HTML}{282C36}
    \definecolor{ansi-red}{HTML}{E75C58}
    \definecolor{ansi-red-intense}{HTML}{B22B31}
    \definecolor{ansi-green}{HTML}{00A250}
    \definecolor{ansi-green-intense}{HTML}{007427}
    \definecolor{ansi-yellow}{HTML}{DDB62B}
    \definecolor{ansi-yellow-intense}{HTML}{B27D12}
    \definecolor{ansi-blue}{HTML}{208FFB}
    \definecolor{ansi-blue-intense}{HTML}{0065CA}
    \definecolor{ansi-magenta}{HTML}{D160C4}
    \definecolor{ansi-magenta-intense}{HTML}{A03196}
    \definecolor{ansi-cyan}{HTML}{60C6C8}
    \definecolor{ansi-cyan-intense}{HTML}{258F8F}
    \definecolor{ansi-white}{HTML}{C5C1B4}
    \definecolor{ansi-white-intense}{HTML}{A1A6B2}

    % commands and environments needed by pandoc snippets
    % extracted from the output of `pandoc -s`
    \providecommand{\tightlist}{%
      \setlength{\itemsep}{0pt}\setlength{\parskip}{0pt}}
    \DefineVerbatimEnvironment{Highlighting}{Verbatim}{commandchars=\\\{\}}
    % Add ',fontsize=\small' for more characters per line
    \newenvironment{Shaded}{}{}
    \newcommand{\KeywordTok}[1]{\textcolor[rgb]{0.00,0.44,0.13}{\textbf{{#1}}}}
    \newcommand{\DataTypeTok}[1]{\textcolor[rgb]{0.56,0.13,0.00}{{#1}}}
    \newcommand{\DecValTok}[1]{\textcolor[rgb]{0.25,0.63,0.44}{{#1}}}
    \newcommand{\BaseNTok}[1]{\textcolor[rgb]{0.25,0.63,0.44}{{#1}}}
    \newcommand{\FloatTok}[1]{\textcolor[rgb]{0.25,0.63,0.44}{{#1}}}
    \newcommand{\CharTok}[1]{\textcolor[rgb]{0.25,0.44,0.63}{{#1}}}
    \newcommand{\StringTok}[1]{\textcolor[rgb]{0.25,0.44,0.63}{{#1}}}
    \newcommand{\CommentTok}[1]{\textcolor[rgb]{0.38,0.63,0.69}{\textit{{#1}}}}
    \newcommand{\OtherTok}[1]{\textcolor[rgb]{0.00,0.44,0.13}{{#1}}}
    \newcommand{\AlertTok}[1]{\textcolor[rgb]{1.00,0.00,0.00}{\textbf{{#1}}}}
    \newcommand{\FunctionTok}[1]{\textcolor[rgb]{0.02,0.16,0.49}{{#1}}}
    \newcommand{\RegionMarkerTok}[1]{{#1}}
    \newcommand{\ErrorTok}[1]{\textcolor[rgb]{1.00,0.00,0.00}{\textbf{{#1}}}}
    \newcommand{\NormalTok}[1]{{#1}}
    
    % Additional commands for more recent versions of Pandoc
    \newcommand{\ConstantTok}[1]{\textcolor[rgb]{0.53,0.00,0.00}{{#1}}}
    \newcommand{\SpecialCharTok}[1]{\textcolor[rgb]{0.25,0.44,0.63}{{#1}}}
    \newcommand{\VerbatimStringTok}[1]{\textcolor[rgb]{0.25,0.44,0.63}{{#1}}}
    \newcommand{\SpecialStringTok}[1]{\textcolor[rgb]{0.73,0.40,0.53}{{#1}}}
    \newcommand{\ImportTok}[1]{{#1}}
    \newcommand{\DocumentationTok}[1]{\textcolor[rgb]{0.73,0.13,0.13}{\textit{{#1}}}}
    \newcommand{\AnnotationTok}[1]{\textcolor[rgb]{0.38,0.63,0.69}{\textbf{\textit{{#1}}}}}
    \newcommand{\CommentVarTok}[1]{\textcolor[rgb]{0.38,0.63,0.69}{\textbf{\textit{{#1}}}}}
    \newcommand{\VariableTok}[1]{\textcolor[rgb]{0.10,0.09,0.49}{{#1}}}
    \newcommand{\ControlFlowTok}[1]{\textcolor[rgb]{0.00,0.44,0.13}{\textbf{{#1}}}}
    \newcommand{\OperatorTok}[1]{\textcolor[rgb]{0.40,0.40,0.40}{{#1}}}
    \newcommand{\BuiltInTok}[1]{{#1}}
    \newcommand{\ExtensionTok}[1]{{#1}}
    \newcommand{\PreprocessorTok}[1]{\textcolor[rgb]{0.74,0.48,0.00}{{#1}}}
    \newcommand{\AttributeTok}[1]{\textcolor[rgb]{0.49,0.56,0.16}{{#1}}}
    \newcommand{\InformationTok}[1]{\textcolor[rgb]{0.38,0.63,0.69}{\textbf{\textit{{#1}}}}}
    \newcommand{\WarningTok}[1]{\textcolor[rgb]{0.38,0.63,0.69}{\textbf{\textit{{#1}}}}}
    
    
    % Define a nice break command that doesn't care if a line doesn't already
    % exist.
    \def\br{\hspace*{\fill} \\* }
    % Math Jax compatability definitions
    \def\gt{>}
    \def\lt{<}
    % Document parameters
    \title{Tensors\_in\_PyTorch\_(Udacity)}
    
    
    

    % Pygments definitions
    
\makeatletter
\def\PY@reset{\let\PY@it=\relax \let\PY@bf=\relax%
    \let\PY@ul=\relax \let\PY@tc=\relax%
    \let\PY@bc=\relax \let\PY@ff=\relax}
\def\PY@tok#1{\csname PY@tok@#1\endcsname}
\def\PY@toks#1+{\ifx\relax#1\empty\else%
    \PY@tok{#1}\expandafter\PY@toks\fi}
\def\PY@do#1{\PY@bc{\PY@tc{\PY@ul{%
    \PY@it{\PY@bf{\PY@ff{#1}}}}}}}
\def\PY#1#2{\PY@reset\PY@toks#1+\relax+\PY@do{#2}}

\expandafter\def\csname PY@tok@w\endcsname{\def\PY@tc##1{\textcolor[rgb]{0.73,0.73,0.73}{##1}}}
\expandafter\def\csname PY@tok@c\endcsname{\let\PY@it=\textit\def\PY@tc##1{\textcolor[rgb]{0.25,0.50,0.50}{##1}}}
\expandafter\def\csname PY@tok@cp\endcsname{\def\PY@tc##1{\textcolor[rgb]{0.74,0.48,0.00}{##1}}}
\expandafter\def\csname PY@tok@k\endcsname{\let\PY@bf=\textbf\def\PY@tc##1{\textcolor[rgb]{0.00,0.50,0.00}{##1}}}
\expandafter\def\csname PY@tok@kp\endcsname{\def\PY@tc##1{\textcolor[rgb]{0.00,0.50,0.00}{##1}}}
\expandafter\def\csname PY@tok@kt\endcsname{\def\PY@tc##1{\textcolor[rgb]{0.69,0.00,0.25}{##1}}}
\expandafter\def\csname PY@tok@o\endcsname{\def\PY@tc##1{\textcolor[rgb]{0.40,0.40,0.40}{##1}}}
\expandafter\def\csname PY@tok@ow\endcsname{\let\PY@bf=\textbf\def\PY@tc##1{\textcolor[rgb]{0.67,0.13,1.00}{##1}}}
\expandafter\def\csname PY@tok@nb\endcsname{\def\PY@tc##1{\textcolor[rgb]{0.00,0.50,0.00}{##1}}}
\expandafter\def\csname PY@tok@nf\endcsname{\def\PY@tc##1{\textcolor[rgb]{0.00,0.00,1.00}{##1}}}
\expandafter\def\csname PY@tok@nc\endcsname{\let\PY@bf=\textbf\def\PY@tc##1{\textcolor[rgb]{0.00,0.00,1.00}{##1}}}
\expandafter\def\csname PY@tok@nn\endcsname{\let\PY@bf=\textbf\def\PY@tc##1{\textcolor[rgb]{0.00,0.00,1.00}{##1}}}
\expandafter\def\csname PY@tok@ne\endcsname{\let\PY@bf=\textbf\def\PY@tc##1{\textcolor[rgb]{0.82,0.25,0.23}{##1}}}
\expandafter\def\csname PY@tok@nv\endcsname{\def\PY@tc##1{\textcolor[rgb]{0.10,0.09,0.49}{##1}}}
\expandafter\def\csname PY@tok@no\endcsname{\def\PY@tc##1{\textcolor[rgb]{0.53,0.00,0.00}{##1}}}
\expandafter\def\csname PY@tok@nl\endcsname{\def\PY@tc##1{\textcolor[rgb]{0.63,0.63,0.00}{##1}}}
\expandafter\def\csname PY@tok@ni\endcsname{\let\PY@bf=\textbf\def\PY@tc##1{\textcolor[rgb]{0.60,0.60,0.60}{##1}}}
\expandafter\def\csname PY@tok@na\endcsname{\def\PY@tc##1{\textcolor[rgb]{0.49,0.56,0.16}{##1}}}
\expandafter\def\csname PY@tok@nt\endcsname{\let\PY@bf=\textbf\def\PY@tc##1{\textcolor[rgb]{0.00,0.50,0.00}{##1}}}
\expandafter\def\csname PY@tok@nd\endcsname{\def\PY@tc##1{\textcolor[rgb]{0.67,0.13,1.00}{##1}}}
\expandafter\def\csname PY@tok@s\endcsname{\def\PY@tc##1{\textcolor[rgb]{0.73,0.13,0.13}{##1}}}
\expandafter\def\csname PY@tok@sd\endcsname{\let\PY@it=\textit\def\PY@tc##1{\textcolor[rgb]{0.73,0.13,0.13}{##1}}}
\expandafter\def\csname PY@tok@si\endcsname{\let\PY@bf=\textbf\def\PY@tc##1{\textcolor[rgb]{0.73,0.40,0.53}{##1}}}
\expandafter\def\csname PY@tok@se\endcsname{\let\PY@bf=\textbf\def\PY@tc##1{\textcolor[rgb]{0.73,0.40,0.13}{##1}}}
\expandafter\def\csname PY@tok@sr\endcsname{\def\PY@tc##1{\textcolor[rgb]{0.73,0.40,0.53}{##1}}}
\expandafter\def\csname PY@tok@ss\endcsname{\def\PY@tc##1{\textcolor[rgb]{0.10,0.09,0.49}{##1}}}
\expandafter\def\csname PY@tok@sx\endcsname{\def\PY@tc##1{\textcolor[rgb]{0.00,0.50,0.00}{##1}}}
\expandafter\def\csname PY@tok@m\endcsname{\def\PY@tc##1{\textcolor[rgb]{0.40,0.40,0.40}{##1}}}
\expandafter\def\csname PY@tok@gh\endcsname{\let\PY@bf=\textbf\def\PY@tc##1{\textcolor[rgb]{0.00,0.00,0.50}{##1}}}
\expandafter\def\csname PY@tok@gu\endcsname{\let\PY@bf=\textbf\def\PY@tc##1{\textcolor[rgb]{0.50,0.00,0.50}{##1}}}
\expandafter\def\csname PY@tok@gd\endcsname{\def\PY@tc##1{\textcolor[rgb]{0.63,0.00,0.00}{##1}}}
\expandafter\def\csname PY@tok@gi\endcsname{\def\PY@tc##1{\textcolor[rgb]{0.00,0.63,0.00}{##1}}}
\expandafter\def\csname PY@tok@gr\endcsname{\def\PY@tc##1{\textcolor[rgb]{1.00,0.00,0.00}{##1}}}
\expandafter\def\csname PY@tok@ge\endcsname{\let\PY@it=\textit}
\expandafter\def\csname PY@tok@gs\endcsname{\let\PY@bf=\textbf}
\expandafter\def\csname PY@tok@gp\endcsname{\let\PY@bf=\textbf\def\PY@tc##1{\textcolor[rgb]{0.00,0.00,0.50}{##1}}}
\expandafter\def\csname PY@tok@go\endcsname{\def\PY@tc##1{\textcolor[rgb]{0.53,0.53,0.53}{##1}}}
\expandafter\def\csname PY@tok@gt\endcsname{\def\PY@tc##1{\textcolor[rgb]{0.00,0.27,0.87}{##1}}}
\expandafter\def\csname PY@tok@err\endcsname{\def\PY@bc##1{\setlength{\fboxsep}{0pt}\fcolorbox[rgb]{1.00,0.00,0.00}{1,1,1}{\strut ##1}}}
\expandafter\def\csname PY@tok@kc\endcsname{\let\PY@bf=\textbf\def\PY@tc##1{\textcolor[rgb]{0.00,0.50,0.00}{##1}}}
\expandafter\def\csname PY@tok@kd\endcsname{\let\PY@bf=\textbf\def\PY@tc##1{\textcolor[rgb]{0.00,0.50,0.00}{##1}}}
\expandafter\def\csname PY@tok@kn\endcsname{\let\PY@bf=\textbf\def\PY@tc##1{\textcolor[rgb]{0.00,0.50,0.00}{##1}}}
\expandafter\def\csname PY@tok@kr\endcsname{\let\PY@bf=\textbf\def\PY@tc##1{\textcolor[rgb]{0.00,0.50,0.00}{##1}}}
\expandafter\def\csname PY@tok@bp\endcsname{\def\PY@tc##1{\textcolor[rgb]{0.00,0.50,0.00}{##1}}}
\expandafter\def\csname PY@tok@fm\endcsname{\def\PY@tc##1{\textcolor[rgb]{0.00,0.00,1.00}{##1}}}
\expandafter\def\csname PY@tok@vc\endcsname{\def\PY@tc##1{\textcolor[rgb]{0.10,0.09,0.49}{##1}}}
\expandafter\def\csname PY@tok@vg\endcsname{\def\PY@tc##1{\textcolor[rgb]{0.10,0.09,0.49}{##1}}}
\expandafter\def\csname PY@tok@vi\endcsname{\def\PY@tc##1{\textcolor[rgb]{0.10,0.09,0.49}{##1}}}
\expandafter\def\csname PY@tok@vm\endcsname{\def\PY@tc##1{\textcolor[rgb]{0.10,0.09,0.49}{##1}}}
\expandafter\def\csname PY@tok@sa\endcsname{\def\PY@tc##1{\textcolor[rgb]{0.73,0.13,0.13}{##1}}}
\expandafter\def\csname PY@tok@sb\endcsname{\def\PY@tc##1{\textcolor[rgb]{0.73,0.13,0.13}{##1}}}
\expandafter\def\csname PY@tok@sc\endcsname{\def\PY@tc##1{\textcolor[rgb]{0.73,0.13,0.13}{##1}}}
\expandafter\def\csname PY@tok@dl\endcsname{\def\PY@tc##1{\textcolor[rgb]{0.73,0.13,0.13}{##1}}}
\expandafter\def\csname PY@tok@s2\endcsname{\def\PY@tc##1{\textcolor[rgb]{0.73,0.13,0.13}{##1}}}
\expandafter\def\csname PY@tok@sh\endcsname{\def\PY@tc##1{\textcolor[rgb]{0.73,0.13,0.13}{##1}}}
\expandafter\def\csname PY@tok@s1\endcsname{\def\PY@tc##1{\textcolor[rgb]{0.73,0.13,0.13}{##1}}}
\expandafter\def\csname PY@tok@mb\endcsname{\def\PY@tc##1{\textcolor[rgb]{0.40,0.40,0.40}{##1}}}
\expandafter\def\csname PY@tok@mf\endcsname{\def\PY@tc##1{\textcolor[rgb]{0.40,0.40,0.40}{##1}}}
\expandafter\def\csname PY@tok@mh\endcsname{\def\PY@tc##1{\textcolor[rgb]{0.40,0.40,0.40}{##1}}}
\expandafter\def\csname PY@tok@mi\endcsname{\def\PY@tc##1{\textcolor[rgb]{0.40,0.40,0.40}{##1}}}
\expandafter\def\csname PY@tok@il\endcsname{\def\PY@tc##1{\textcolor[rgb]{0.40,0.40,0.40}{##1}}}
\expandafter\def\csname PY@tok@mo\endcsname{\def\PY@tc##1{\textcolor[rgb]{0.40,0.40,0.40}{##1}}}
\expandafter\def\csname PY@tok@ch\endcsname{\let\PY@it=\textit\def\PY@tc##1{\textcolor[rgb]{0.25,0.50,0.50}{##1}}}
\expandafter\def\csname PY@tok@cm\endcsname{\let\PY@it=\textit\def\PY@tc##1{\textcolor[rgb]{0.25,0.50,0.50}{##1}}}
\expandafter\def\csname PY@tok@cpf\endcsname{\let\PY@it=\textit\def\PY@tc##1{\textcolor[rgb]{0.25,0.50,0.50}{##1}}}
\expandafter\def\csname PY@tok@c1\endcsname{\let\PY@it=\textit\def\PY@tc##1{\textcolor[rgb]{0.25,0.50,0.50}{##1}}}
\expandafter\def\csname PY@tok@cs\endcsname{\let\PY@it=\textit\def\PY@tc##1{\textcolor[rgb]{0.25,0.50,0.50}{##1}}}

\def\PYZbs{\char`\\}
\def\PYZus{\char`\_}
\def\PYZob{\char`\{}
\def\PYZcb{\char`\}}
\def\PYZca{\char`\^}
\def\PYZam{\char`\&}
\def\PYZlt{\char`\<}
\def\PYZgt{\char`\>}
\def\PYZsh{\char`\#}
\def\PYZpc{\char`\%}
\def\PYZdl{\char`\$}
\def\PYZhy{\char`\-}
\def\PYZsq{\char`\'}
\def\PYZdq{\char`\"}
\def\PYZti{\char`\~}
% for compatibility with earlier versions
\def\PYZat{@}
\def\PYZlb{[}
\def\PYZrb{]}
\makeatother


    % Exact colors from NB
    \definecolor{incolor}{rgb}{0.0, 0.0, 0.5}
    \definecolor{outcolor}{rgb}{0.545, 0.0, 0.0}



    
    % Prevent overflowing lines due to hard-to-break entities
    \sloppy 
    % Setup hyperref package
    \hypersetup{
      breaklinks=true,  % so long urls are correctly broken across lines
      colorlinks=true,
      urlcolor=urlcolor,
      linkcolor=linkcolor,
      citecolor=citecolor,
      }
    % Slightly bigger margins than the latex defaults
    
    \geometry{verbose,tmargin=1in,bmargin=1in,lmargin=1in,rmargin=1in}
    
    

    \begin{document}
    
    
    \maketitle
    
    

    
    

    \section{Introduction to Deep Learning with
PyTorch}\label{introduction-to-deep-learning-with-pytorch}

In this notebook, you'll get introduced to
\href{http://pytorch.org/}{PyTorch}, a framework for building and
training neural networks. PyTorch in a lot of ways behaves like the
arrays you love from Numpy. These Numpy arrays, after all, are just
tensors. PyTorch takes these tensors and makes it simple to move them to
GPUs for the faster processing needed when training neural networks. It
also provides a module that automatically calculates gradients (for
backpropagation!) and another module specifically for building neural
networks. All together, PyTorch ends up being more coherent with Python
and the Numpy/Scipy stack compared to TensorFlow and other frameworks.

    \subsection{Neural Networks}\label{neural-networks}

Deep Learning is based on artificial neural networks which have been
around in some form since the late 1950s. The networks are built from
individual parts approximating neurons, typically called units or simply
"neurons." Each unit has some number of weighted inputs. These weighted
inputs are summed together (a linear combination) then passed through an
activation function to get the unit's output.

Mathematically this looks like:

\[
\begin{align}
y &= f(w_1 x_1 + w_2 x_2 + b) \\
y &= f\left(\sum_i w_i x_i +b \right)
\end{align}
\]

With vectors this is the dot/inner product of two vectors:

\[
h = \begin{bmatrix}
x_1 \, x_2 \cdots  x_n
\end{bmatrix}
\cdot 
\begin{bmatrix}
           w_1 \\
           w_2 \\
           \vdots \\
           w_n
\end{bmatrix}
\]

    \subsection{Tensors}\label{tensors}

It turns out neural network computations are just a bunch of linear
algebra operations on \emph{tensors}, a generalization of matrices. A
vector is a 1-dimensional tensor, a matrix is a 2-dimensional tensor, an
array with three indices is a 3-dimensional tensor (RGB color images for
example). The fundamental data structure for neural networks are tensors
and PyTorch (as well as pretty much every other deep learning framework)
is built around tensors.

With the basics covered, it's time to explore how we can use PyTorch to
build a simple neural network.

    \begin{Verbatim}[commandchars=\\\{\}]
{\color{incolor}In [{\color{incolor}0}]:} \PY{c+c1}{\PYZsh{} First, import PyTorch}
        \PY{k+kn}{import} \PY{n+nn}{torch}
\end{Verbatim}


    \begin{Verbatim}[commandchars=\\\{\}]
{\color{incolor}In [{\color{incolor}0}]:} \PY{k}{def} \PY{n+nf}{activation}\PY{p}{(}\PY{n}{x}\PY{p}{)}\PY{p}{:}
            \PY{l+s+sd}{\PYZdq{}\PYZdq{}\PYZdq{} Sigmoid activation function }
        \PY{l+s+sd}{    }
        \PY{l+s+sd}{        Arguments}
        \PY{l+s+sd}{        \PYZhy{}\PYZhy{}\PYZhy{}\PYZhy{}\PYZhy{}\PYZhy{}\PYZhy{}\PYZhy{}\PYZhy{}}
        \PY{l+s+sd}{        x: torch.Tensor}
        \PY{l+s+sd}{    \PYZdq{}\PYZdq{}\PYZdq{}}
            \PY{k}{return} \PY{l+m+mi}{1}\PY{o}{/}\PY{p}{(}\PY{l+m+mi}{1}\PY{o}{+}\PY{n}{torch}\PY{o}{.}\PY{n}{exp}\PY{p}{(}\PY{o}{\PYZhy{}}\PY{n}{x}\PY{p}{)}\PY{p}{)}
\end{Verbatim}


    \begin{Verbatim}[commandchars=\\\{\}]
{\color{incolor}In [{\color{incolor}0}]:} \PY{c+c1}{\PYZsh{}\PYZsh{}\PYZsh{} Generate some data}
        \PY{n}{torch}\PY{o}{.}\PY{n}{manual\PYZus{}seed}\PY{p}{(}\PY{l+m+mi}{7}\PY{p}{)} \PY{c+c1}{\PYZsh{} Set the random seed so things are predictable}
        
        \PY{c+c1}{\PYZsh{} Features are 3 random normal variables}
        \PY{n}{features} \PY{o}{=} \PY{n}{torch}\PY{o}{.}\PY{n}{randn}\PY{p}{(}\PY{p}{(}\PY{l+m+mi}{1}\PY{p}{,} \PY{l+m+mi}{5}\PY{p}{)}\PY{p}{)}
        \PY{c+c1}{\PYZsh{} True weights for our data, random normal variables again}
        \PY{n}{weights} \PY{o}{=} \PY{n}{torch}\PY{o}{.}\PY{n}{randn\PYZus{}like}\PY{p}{(}\PY{n}{features}\PY{p}{)}
        \PY{c+c1}{\PYZsh{} and a true bias term}
        \PY{n}{bias} \PY{o}{=} \PY{n}{torch}\PY{o}{.}\PY{n}{randn}\PY{p}{(}\PY{p}{(}\PY{l+m+mi}{1}\PY{p}{,} \PY{l+m+mi}{1}\PY{p}{)}\PY{p}{)}
\end{Verbatim}


    Above I generated data we can use to get the output of our simple
network. This is all just random for now, going forward we'll start
using normal data. Going through each relevant line:

\texttt{features\ =\ torch.randn((1,\ 5))} creates a tensor with shape
\texttt{(1,\ 5)}, one row and five columns, that contains values
randomly distributed according to the normal distribution with a mean of
zero and standard deviation of one.

\texttt{weights\ =\ torch.randn\_like(features)} creates another tensor
with the same shape as \texttt{features}, again containing values from a
normal distribution.

Finally, \texttt{bias\ =\ torch.randn((1,\ 1))} creates a single value
from a normal distribution.

PyTorch tensors can be added, multiplied, subtracted, etc, just like
Numpy arrays. In general, you'll use PyTorch tensors pretty much the
same way you'd use Numpy arrays. They come with some nice benefits
though such as GPU acceleration which we'll get to later. For now, use
the generated data to calculate the output of this simple single layer
network. \textgreater{} \textbf{Exercise}: Calculate the output of the
network with input features \texttt{features}, weights \texttt{weights},
and bias \texttt{bias}. Similar to Numpy, PyTorch has a
\href{https://pytorch.org/docs/stable/torch.html\#torch.sum}{\texttt{torch.sum()}}
function, as well as a \texttt{.sum()} method on tensors, for taking
sums. Use the function \texttt{activation} defined above as the
activation function.

    \begin{Verbatim}[commandchars=\\\{\}]
{\color{incolor}In [{\color{incolor}0}]:} \PY{c+c1}{\PYZsh{}\PYZsh{}\PYZsh{} Solution}
        
        \PY{c+c1}{\PYZsh{} Now, make our labels from our data and true weights}
        
        \PY{n}{y} \PY{o}{=} \PY{n}{activation}\PY{p}{(}\PY{n}{torch}\PY{o}{.}\PY{n}{sum}\PY{p}{(}\PY{n}{features} \PY{o}{*} \PY{n}{weights}\PY{p}{)} \PY{o}{+} \PY{n}{bias}\PY{p}{)}
        \PY{n}{y} \PY{o}{=} \PY{n}{activation}\PY{p}{(}\PY{p}{(}\PY{n}{features} \PY{o}{*} \PY{n}{weights}\PY{p}{)}\PY{o}{.}\PY{n}{sum}\PY{p}{(}\PY{p}{)} \PY{o}{+} \PY{n}{bias}\PY{p}{)}
\end{Verbatim}


    You can do the multiplication and sum in the same operation using a
matrix multiplication. In general, you'll want to use matrix
multiplications since they are more efficient and accelerated using
modern libraries and high-performance computing on GPUs.

Here, we want to do a matrix multiplication of the features and the
weights. For this we can use
\href{https://pytorch.org/docs/stable/torch.html\#torch.mm}{\texttt{torch.mm()}}
or
\href{https://pytorch.org/docs/stable/torch.html\#torch.matmul}{\texttt{torch.matmul()}}
which is somewhat more complicated and supports broadcasting. If we try
to do it with \texttt{features} and \texttt{weights} as they are, we'll
get an error

\begin{Shaded}
\begin{Highlighting}[]
\OperatorTok{>>}\NormalTok{ torch.mm(features, weights)}

\OperatorTok{---------------------------------------------------------------------------}
\PreprocessorTok{RuntimeError}\NormalTok{                              Traceback (most recent call last)}
\OperatorTok{<}\NormalTok{ipython}\OperatorTok{-}\BuiltInTok{input}\OperatorTok{-}\DecValTok{13}\OperatorTok{-}\NormalTok{15d592eb5279}\OperatorTok{>} \KeywordTok{in} \OperatorTok{<}\NormalTok{module}\OperatorTok{>}\NormalTok{()}
\OperatorTok{---->} \DecValTok{1}\NormalTok{ torch.mm(features, weights)}

\PreprocessorTok{RuntimeError}\NormalTok{: size mismatch, m1: [}\DecValTok{1}\NormalTok{ x }\DecValTok{5}\NormalTok{], m2: [}\DecValTok{1}\NormalTok{ x }\DecValTok{5}\NormalTok{] at }\OperatorTok{/}\NormalTok{Users}\OperatorTok{/}\NormalTok{soumith}\OperatorTok{/}\NormalTok{minicondabuild3}\OperatorTok{/}\NormalTok{conda}\OperatorTok{-}\NormalTok{bld}\OperatorTok{/}\NormalTok{pytorch_1524590658547}\OperatorTok{/}\NormalTok{work}\OperatorTok{/}\NormalTok{aten}\OperatorTok{/}\NormalTok{src}\OperatorTok{/}\NormalTok{TH}\OperatorTok{/}\NormalTok{generic}\OperatorTok{/}\NormalTok{THTensorMath.c:}\DecValTok{2033}
\end{Highlighting}
\end{Shaded}

As you're building neural networks in any framework, you'll see this
often. Really often. What's happening here is our tensors aren't the
correct shapes to perform a matrix multiplication. Remember that for
matrix multiplications, the number of columns in the first tensor must
equal to the number of rows in the second column. Both \texttt{features}
and \texttt{weights} have the same shape, \texttt{(1,\ 5)}. This means
we need to change the shape of \texttt{weights} to get the matrix
multiplication to work.

\textbf{Note:} To see the shape of a tensor called \texttt{tensor}, use
\texttt{tensor.shape}. If you're building neural networks, you'll be
using this method often.

There are a few options here:
\href{https://pytorch.org/docs/stable/tensors.html\#torch.Tensor.reshape}{\texttt{weights.reshape()}},
\href{https://pytorch.org/docs/stable/tensors.html\#torch.Tensor.resize_}{\texttt{weights.resize\_()}},
and
\href{https://pytorch.org/docs/stable/tensors.html\#torch.Tensor.view}{\texttt{weights.view()}}.

\begin{itemize}
\tightlist
\item
  \texttt{weights.reshape(a,\ b)} will return a new tensor with the same
  data as \texttt{weights} with size \texttt{(a,\ b)} sometimes, and
  sometimes a clone, as in it copies the data to another part of memory.
\item
  \texttt{weights.resize\_(a,\ b)} returns the same tensor with a
  different shape. However, if the new shape results in fewer elements
  than the original tensor, some elements will be removed from the
  tensor (but not from memory). If the new shape results in more
  elements than the original tensor, new elements will be uninitialized
  in memory. Here I should note that the underscore at the end of the
  method denotes that this method is performed \textbf{in-place}. Here
  is a great forum thread to
  \href{https://discuss.pytorch.org/t/what-is-in-place-operation/16244}{read
  more about in-place operations} in PyTorch.
\item
  \texttt{weights.view(a,\ b)} will return a new tensor with the same
  data as \texttt{weights} with size \texttt{(a,\ b)}.
\end{itemize}

I usually use \texttt{.view()}, but any of the three methods will work
for this. So, now we can reshape \texttt{weights} to have five rows and
one column with something like \texttt{weights.view(5,\ 1)}.

\begin{quote}
\textbf{Exercise}: Calculate the output of our little network using
matrix multiplication.
\end{quote}

    \begin{Verbatim}[commandchars=\\\{\}]
{\color{incolor}In [{\color{incolor}0}]:} \PY{c+c1}{\PYZsh{}\PYZsh{} Solution}
        
        \PY{n}{y} \PY{o}{=} \PY{n}{activation}\PY{p}{(}\PY{n}{torch}\PY{o}{.}\PY{n}{mm}\PY{p}{(}\PY{n}{features}\PY{p}{,} \PY{n}{weights}\PY{o}{.}\PY{n}{view}\PY{p}{(}\PY{l+m+mi}{5}\PY{p}{,}\PY{l+m+mi}{1}\PY{p}{)}\PY{p}{)} \PY{o}{+} \PY{n}{bias}\PY{p}{)}
\end{Verbatim}


    \subsubsection{Stack them up!}\label{stack-them-up}

That's how you can calculate the output for a single neuron. The real
power of this algorithm happens when you start stacking these individual
units into layers and stacks of layers, into a network of neurons. The
output of one layer of neurons becomes the input for the next layer.
With multiple input units and output units, we now need to express the
weights as a matrix.

The first layer shown on the bottom here are the inputs, understandably
called the \textbf{input layer}. The middle layer is called the
\textbf{hidden layer}, and the final layer (on the right) is the
\textbf{output layer}. We can express this network mathematically with
matrices again and use matrix multiplication to get linear combinations
for each unit in one operation. For example, the hidden layer (\(h_1\)
and \(h_2\) here) can be calculated

\[
\vec{h} = [h_1 \, h_2] = 
\begin{bmatrix}
x_1 \, x_2 \cdots \, x_n
\end{bmatrix}
\cdot 
\begin{bmatrix}
           w_{11} & w_{12} \\
           w_{21} &w_{22} \\
           \vdots &\vdots \\
           w_{n1} &w_{n2}
\end{bmatrix}
\]

The output for this small network is found by treating the hidden layer
as inputs for the output unit. The network output is expressed simply

\[
y =  f_2 \! \left(\, f_1 \! \left(\vec{x} \, \mathbf{W_1}\right) \mathbf{W_2} \right)
\]

    \begin{Verbatim}[commandchars=\\\{\}]
{\color{incolor}In [{\color{incolor}0}]:} \PY{c+c1}{\PYZsh{}\PYZsh{}\PYZsh{} Generate some data}
        \PY{n}{torch}\PY{o}{.}\PY{n}{manual\PYZus{}seed}\PY{p}{(}\PY{l+m+mi}{7}\PY{p}{)} \PY{c+c1}{\PYZsh{} Set the random seed so things are predictable}
        
        \PY{c+c1}{\PYZsh{} Features are 3 random normal variables}
        \PY{n}{features} \PY{o}{=} \PY{n}{torch}\PY{o}{.}\PY{n}{randn}\PY{p}{(}\PY{p}{(}\PY{l+m+mi}{1}\PY{p}{,} \PY{l+m+mi}{3}\PY{p}{)}\PY{p}{)}
        
        \PY{c+c1}{\PYZsh{} Define the size of each layer in our network}
        \PY{n}{n\PYZus{}input} \PY{o}{=} \PY{n}{features}\PY{o}{.}\PY{n}{shape}\PY{p}{[}\PY{l+m+mi}{1}\PY{p}{]}     \PY{c+c1}{\PYZsh{} Number of input units, must match number of input features}
        \PY{n}{n\PYZus{}hidden} \PY{o}{=} \PY{l+m+mi}{2}                    \PY{c+c1}{\PYZsh{} Number of hidden units }
        \PY{n}{n\PYZus{}output} \PY{o}{=} \PY{l+m+mi}{1}                    \PY{c+c1}{\PYZsh{} Number of output units}
        
        \PY{c+c1}{\PYZsh{} Weights for inputs to hidden layer}
        \PY{n}{W1} \PY{o}{=} \PY{n}{torch}\PY{o}{.}\PY{n}{randn}\PY{p}{(}\PY{n}{n\PYZus{}input}\PY{p}{,} \PY{n}{n\PYZus{}hidden}\PY{p}{)}
        \PY{c+c1}{\PYZsh{} Weights for hidden layer to output layer}
        \PY{n}{W2} \PY{o}{=} \PY{n}{torch}\PY{o}{.}\PY{n}{randn}\PY{p}{(}\PY{n}{n\PYZus{}hidden}\PY{p}{,} \PY{n}{n\PYZus{}output}\PY{p}{)}
        
        \PY{c+c1}{\PYZsh{} and bias terms for hidden and output layers}
        \PY{n}{B1} \PY{o}{=} \PY{n}{torch}\PY{o}{.}\PY{n}{randn}\PY{p}{(}\PY{p}{(}\PY{l+m+mi}{1}\PY{p}{,} \PY{n}{n\PYZus{}hidden}\PY{p}{)}\PY{p}{)}
        \PY{n}{B2} \PY{o}{=} \PY{n}{torch}\PY{o}{.}\PY{n}{randn}\PY{p}{(}\PY{p}{(}\PY{l+m+mi}{1}\PY{p}{,} \PY{n}{n\PYZus{}output}\PY{p}{)}\PY{p}{)}
\end{Verbatim}


    \begin{quote}
\textbf{Exercise:} Calculate the output for this multi-layer network
using the weights \texttt{W1} \& \texttt{W2}, and the biases,
\texttt{B1} \& \texttt{B2}.
\end{quote}

    \begin{Verbatim}[commandchars=\\\{\}]
{\color{incolor}In [{\color{incolor}0}]:} \PY{c+c1}{\PYZsh{}\PYZsh{}\PYZsh{} Solution}
        
        \PY{n}{h} \PY{o}{=} \PY{n}{activation}\PY{p}{(}\PY{n}{torch}\PY{o}{.}\PY{n}{mm}\PY{p}{(}\PY{n}{features}\PY{p}{,} \PY{n}{W1}\PY{p}{)} \PY{o}{+} \PY{n}{B1}\PY{p}{)}
        \PY{n}{output} \PY{o}{=} \PY{n}{activation}\PY{p}{(}\PY{n}{torch}\PY{o}{.}\PY{n}{mm}\PY{p}{(}\PY{n}{h}\PY{p}{,} \PY{n}{W2}\PY{p}{)} \PY{o}{+} \PY{n}{B2}\PY{p}{)}
        \PY{n+nb}{print}\PY{p}{(}\PY{n}{output}\PY{p}{)}
\end{Verbatim}


    \begin{Verbatim}[commandchars=\\\{\}]
tensor([[ 0.3171]])

    \end{Verbatim}

    If you did this correctly, you should see the output
\texttt{tensor({[}{[}\ 0.3171{]}{]})}.

The number of hidden units a parameter of the network, often called a
\textbf{hyperparameter} to differentiate it from the weights and biases
parameters. As you'll see later when we discuss training a neural
network, the more hidden units a network has, and the more layers, the
better able it is to learn from data and make accurate predictions.

    \subsection{Numpy to Torch and back}\label{numpy-to-torch-and-back}

Special bonus section! PyTorch has a great feature for converting
between Numpy arrays and Torch tensors. To create a tensor from a Numpy
array, use \texttt{torch.from\_numpy()}. To convert a tensor to a Numpy
array, use the \texttt{.numpy()} method.

    \begin{Verbatim}[commandchars=\\\{\}]
{\color{incolor}In [{\color{incolor}0}]:} \PY{k+kn}{import} \PY{n+nn}{numpy} \PY{k}{as} \PY{n+nn}{np}
        \PY{n}{a} \PY{o}{=} \PY{n}{np}\PY{o}{.}\PY{n}{random}\PY{o}{.}\PY{n}{rand}\PY{p}{(}\PY{l+m+mi}{4}\PY{p}{,}\PY{l+m+mi}{3}\PY{p}{)}
        \PY{n}{a}
\end{Verbatim}


\begin{Verbatim}[commandchars=\\\{\}]
{\color{outcolor}Out[{\color{outcolor}0}]:} array([[ 0.33669496,  0.59531562,  0.65433944],
               [ 0.86531224,  0.59945364,  0.28043973],
               [ 0.48409303,  0.98357622,  0.33884284],
               [ 0.25591391,  0.51081783,  0.39986403]])
\end{Verbatim}
            
    \begin{Verbatim}[commandchars=\\\{\}]
{\color{incolor}In [{\color{incolor}0}]:} \PY{n}{b} \PY{o}{=} \PY{n}{torch}\PY{o}{.}\PY{n}{from\PYZus{}numpy}\PY{p}{(}\PY{n}{a}\PY{p}{)}
        \PY{n}{b}
\end{Verbatim}


\begin{Verbatim}[commandchars=\\\{\}]
{\color{outcolor}Out[{\color{outcolor}0}]:} 
         0.3367  0.5953  0.6543
         0.8653  0.5995  0.2804
         0.4841  0.9836  0.3388
         0.2559  0.5108  0.3999
        [torch.DoubleTensor of size 4x3]
\end{Verbatim}
            
    \begin{Verbatim}[commandchars=\\\{\}]
{\color{incolor}In [{\color{incolor}0}]:} \PY{n}{b}\PY{o}{.}\PY{n}{numpy}\PY{p}{(}\PY{p}{)}
\end{Verbatim}


\begin{Verbatim}[commandchars=\\\{\}]
{\color{outcolor}Out[{\color{outcolor}0}]:} array([[ 0.33669496,  0.59531562,  0.65433944],
               [ 0.86531224,  0.59945364,  0.28043973],
               [ 0.48409303,  0.98357622,  0.33884284],
               [ 0.25591391,  0.51081783,  0.39986403]])
\end{Verbatim}
            
    The memory is shared between the Numpy array and Torch tensor, so if you
change the values in-place of one object, the other will change as well.

    \begin{Verbatim}[commandchars=\\\{\}]
{\color{incolor}In [{\color{incolor}0}]:} \PY{c+c1}{\PYZsh{} Multiply PyTorch Tensor by 2, in place}
        \PY{n}{b}\PY{o}{.}\PY{n}{mul\PYZus{}}\PY{p}{(}\PY{l+m+mi}{2}\PY{p}{)}
\end{Verbatim}


\begin{Verbatim}[commandchars=\\\{\}]
{\color{outcolor}Out[{\color{outcolor}0}]:} 
         0.6734  1.1906  1.3087
         1.7306  1.1989  0.5609
         0.9682  1.9672  0.6777
         0.5118  1.0216  0.7997
        [torch.DoubleTensor of size 4x3]
\end{Verbatim}
            
    \begin{Verbatim}[commandchars=\\\{\}]
{\color{incolor}In [{\color{incolor}0}]:} \PY{c+c1}{\PYZsh{} Numpy array matches new values from Tensor}
        \PY{n}{a}
\end{Verbatim}


\begin{Verbatim}[commandchars=\\\{\}]
{\color{outcolor}Out[{\color{outcolor}0}]:} array([[ 0.67338991,  1.19063124,  1.30867888],
               [ 1.73062448,  1.19890728,  0.56087946],
               [ 0.96818606,  1.96715243,  0.67768568],
               [ 0.51182782,  1.02163565,  0.79972807]])
\end{Verbatim}
            

    % Add a bibliography block to the postdoc
    
    
    
    \end{document}
